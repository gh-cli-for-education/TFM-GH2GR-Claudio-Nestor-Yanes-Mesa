\newpage 
\thispagestyle{empty}

\begin{abstract}
{\em
Este trabajo investiga el creciente interés por la gamificación como método para dinamizar el aprendizaje dentro del campo de las Tecnologías de la Información (TI). Presentamos GH2GR, un programa de software que actúa como puente entre GitHub Classroom, una plataforma de enseñanza prevalente para asignaturas de TI, y GoRace, una plataforma de gamificación extensible. Mientras que los enfoques tradicionales de gamificación pueden sin duda ayudar a mejorar el compromiso de los estudiantes, GoRace proporciona un conjunto de herramientas más rico en características y adaptable. GH2GR agiliza la incorporación de las mecánicas de GoRace en los flujos de trabajo existentes de GitHub Classroom. Esta innovación permite a los profesores gamificar sus cursos de TI sin necesidad de utilizar herramientas adicionales. Aprovechando GH2GR, los instructores pueden introducir el poder motivacional de la gamificación dentro del entorno familiar y establecido de GitHub Classroom, minimizando la interrupción de las prácticas de enseñanza actuales. Este artículo explora el diseño y el desarrollo de GH2GR, evaluando su potencial para mejorar la participación de los estudiantes y los resultados del aprendizaje en la enseñanza de TI.
}



\begin{palabrasClave}
Ludificación, Gamificación, GoRace, Github Classroom
\end{palabrasClave}

\end{abstract}
%%%%%%%%%%%%%%%%%%%%%%%%%%%%%%%%%%%%%%%%%%%%%%%%%%%%%%%%%
\newpage 
\vspace*{200px}
\thispagestyle{empty}

\begin{summary}
{
This work investigates the growing interest in gamification as a method to invigorate learning within the Information Technology (IT) field. We introduce GH2GR, a software program that acts as a bridge between GitHub Classroom, a prevalent teaching platform for IT subjects, and GoRace, an extensible gamification platform. While traditional gamification approaches can undoubtedly enhance student engagement, GoRace provides a more feature-rich and adaptable toolkit. GH2GR streamlines the incorporation of GoRace mechanics into existing GitHub Classroom workflows. This innovation empowers professors to seamlessly gamify their IT courses without requiring extensive use of additional tools. By leveraging GH2GR, instructors can introduce the motivational power of gamification within the familiar and established environment of GitHub Classroom, minimizing disruption to current teaching practices. This thesis explores the design and development of GH2GR, evaluating its potential to improve student engagement and learning outcomes in IT education.
}

\em
\begin {keywords}
Gamification, GoRace, Github Classroom
\end {keywords}

\end{summary}