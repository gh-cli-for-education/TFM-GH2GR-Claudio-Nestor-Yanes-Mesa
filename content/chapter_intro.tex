\setlength{\parskip}{1em}

\section{Introducción}
En el panorama en rápida evolución de la enseñanza de las tecnologías de la información (\acrshort{TI}), la ludificación o gamificación ha surgido como una poderosa estrategia para potenciar el aprendizaje mediante la incorporación de elementos de diseño de juegos en contextos no lúdicos\cite{10.1145/3231709}. En este proyecto, nuestro objetivo es proporcionar una alternativa que simplifique el desarrollo de experiencias de ludificación para la enseñanza en el sector \acrshort{TI}. Para este fin, nos valdremos de dos herramientas preexistentes: GoRace y Github Classroom.

GoRace está diseñada para fomentar un nivel más profundo de compromiso y motivación entre los estudiantes proporcionando un entorno flexible y dinámico que se adapta a diversas necesidades educativas. A diferencia de las plataformas tradicionales, GoRace ofrece un conjunto de herramientas que permiten a los educadores crear una experiencia de aprendizaje más envolvente que sea a la vez divertida y eficaz.

Paralelamente, GitHub Classroom se ha consolidado como una plataforma sólida dentro del ecosistema de GitHub, ampliamente adoptada por los educadores para difundir cursos de \acrshort{TI} y gestionar las entregas de tareas de los estudiantes. Su integración con la plataforma GitHub aprovecha herramientas y flujos de trabajo familiares, lo que la convierte en un activo inestimable para la educación en \acrshort{TI}.

Para salvar la brecha entre estas dos plataformas, este trabajo presenta GH2GR, una novedosa solución de software que integra a la perfección GitHub Classroom con GoRace. GH2GR automatiza el proceso de gamificación dentro de GoRace para las asignaturas de \acrshort{TI} que usan GitHub, minimizando las herramientas adicionales requeridas por los educadores. Esta integración permite a los profesores seguir utilizando las herramientas a las que están acostumbrados mientras aprovechan los beneficios de la gamificación a través de GoRace.

Las secciones siguientes profundizarán en los fundamentos teóricos de la gamificación en la educación, la arquitectura y las características de GH2GR, el papel de GitHub Classroom en la educación de \acrshort{TI}, y la funcionalidad y el impacto real de GH2GR.

\section{Gamificación}
La ludificación (o gamificación) es la incorporación estratégica de elementos y principios de diseño de juegos en contextos no lúdicos con la intención de mejoras la experiencia de usuario y el compromiso de estos.\cite{10.1145/1979742.1979575}

En el contexto educativo, la gamificación puede incrementar notablemente el compromiso y la motivación de los alumnos. Al convertir las tareas de aprendizaje en experiencias lúdicas, se favorece la participación activa de los estudiantes y una mejor retención de la información. Además, este método promueve la sensación de logro y avance a través de sistemas de recompensas y reconocimientos.\cite{Tu2014}\cite{Barata2013}\cite{Lister2015}

\section{Github}
El desarrollo de software es un proceso iterativo, caracterizado por el perfeccionamiento y la adaptación constantes.  En este proceso es fundamental el control de versiones, que permite a los desarrolladores hacer un seguimiento de los cambios realizados en el código, volver a versiones anteriores si es necesario y colaborar eficazmente en los proyectos.  Una de las herramientas más populares para el control de versiones es Git, un sistema de control de versiones distribuido (DVCS) que permite trabajar sin conexión y fusionar fácilmente el código de múltiples colaboradores.\cite{gitlabWhatVersion}

En este contexto, GitHub surge como una plataforma basada en la nube que aprovecha las funcionalidades de Git.  GitHub ofrece una interfaz fácil de usar para que los desarrolladores almacenen sus repositorios de código, realicen un seguimiento de los cambios y colaboren en los proyectos.  Estas funcionalidades van más allá del simple control de versiones y ofrecen funciones como el seguimiento de incidencias, herramientas de gestión de proyectos y wikis para la documentación del código, todo ello centralizado en una única plataforma.  Esto fomenta un entorno de colaboración en el que los desarrolladores pueden compartir código, hacer un seguimiento de los errores y trabajar juntos sin problemas.\cite{githubAboutGitHub}

\section{Github Clasroom}

GitHub Classroom es una herramienta de enseñanza diseñada para facilitar la gestión y organización de aulas digitales y deberes. Permite crear tareas para estudiantes que pueden ser  individuales o de grupo, establecer plazos de entrega y monitorear las asignaciones desde el panel del profesor. Además, GitHub Classroom ofrece múltiples funciones que simplifican tareas como dar retroalimentación, evaluar trabajos e integrar herramientas educativas existentes\cite{githubAboutClassroom}.

Con GitHub Classroom, se pueden establecer pruebas que califiquen automáticamente el trabajo de los estudiantes cada vez que ellos actualicen su repositorio de la asignación. Cuando se habilita la autoevaluación, GitHub Actions\cite{githubUnderstandingActions} ejecuta los comandos para su prueba de autoevaluación en un entorno Linux que contiene el código más reciente del estudiante. GitHub Classroom crea los flujos de trabajo necesarios para GitHub Actions. Por lo que no es necesario saber trabajar con estas\cite{githubAutogradingClasroom}.

Github Clasroom expone un cómodo panel de control, donde el profesorado puede configurar pruebas utilizando un \textit{framework} de pruebas, ejecutando un comando personalizado, escribiendo pruebas de entrada/salida o combinar diferentes métodos de prueba\cite{githubAutogradingClasroom}.

\section{GitHub API y GitHub API Classroom}
La \acrshort{API} de GitHub es un conjunto de herramientas y protocolos que permite a los desarrolladores interactuar con GitHub mediante programación. En esencia, proporciona una forma para que las aplicaciones de software se comuniquen con los servidores de GitHub y realicen diversas operaciones, como recuperar información del repositorio, gestionar incidencias y pull requests, e interactuar con cuentas de usuario.\cite{githubAPI}

El propósito de la \acrshort{API} de GitHub es permitir a los desarrolladores crear aplicaciones e integraciones que aprovechen la funcionalidad y los datos de GitHub. Esto puede abarcar desde la automatización de tareas rutinarias, como la creación y gestión de repositorios, hasta la creación de herramientas sofisticadas que analicen el código, realicen un seguimiento del progreso del proyecto o integren GitHub con otros servicios.

Además, GitHub ofrece una \acrshort{API} que facilita la comunicación con su herramienta GitHub Classroom. A diferencia de la \acrshort{API} estándar de GitHub, la \acrshort{API} de Classroom no permite modificar los datos de Classroom, pero sí proporciona acceso a la consulta de una amplia gama de información, como las tareas de una clase, las entregas realizadas y las calificaciones de cada entrega.\cite{ClassroomAPI}

\section{GitHub Apps}
Aunque GitHub destaca en el control de versiones y el fomento de la colaboración, sus capacidades pueden mejorarse aún más a través de GitHub Apps. Se trata de herramientas personalizadas que se integran a la perfección con las funcionalidades de la plataforma a través de \acrshort{API} y \textit{webhooks}. Las  GitHub Apps funciona con un sistema de permisos de grano fino. Esto permite a los usuarios controlar meticulosamente los datos y acciones accesibles a la aplicación, reforzando la seguridad y la confianza del usuario. Además, GitHub Apps utiliza \textit{tokens} de corta duración, lo que minimiza la superficie potencial de ataque.\cite{githubAboutApps}

Las funcionalidades de GitHub Apps son amplias. Pueden operar dentro de la propia plataforma, automatizando tareas como abrir issues, comentar \textit{pull requests} y gestionar proyectos. Y lo que es más importante, las GitHub Apps pueden reaccionar a eventos que ocurren en la plataforma y desencadenar acciones fuera de ella. Esto abre las puertas a una amplia gama de integraciones y personalizaciones, fomentando un flujo de trabajo de desarrollo más ágil.

\section{GitHub Actions}
Aprovechando la ubicuidad de GitHub en el panorama del desarrollo de software, GitHub Actions ofrece un sólido motor de automatización del flujo de trabajo integrado directamente en la plataforma. Esto elimina la necesidad de servidores \acrshort{CI/CD} externos o integraciones complejas. Los desarrolladores pueden definir flujos de trabajo personalizados activados por eventos como un push a una rama o la creación de un \textit{pull request}. Estos flujos de trabajo orquestan una serie de acciones pre-construidas o scripts personalizados, automatizando tareas que antes eran manuales.  Esto puede abarcar la construcción y prueba de código, la racionalización de los despliegues, e incluso la incorporación de análisis de revisión de código.\cite{githubUnderstandingActions}

El verdadero poder de GitHub Actions reside en su extensibilidad.  Existe un rico ecosistema de acciones pre-construidas, que cubren una amplia gama de funcionalidades - desde la construcción y pruebas con frameworks populares hasta el escaneo de seguridad y despliegue en varias plataformas en la nube.  Esta amplia biblioteca permite a los desarrolladores elegir las acciones que mejor se adapten a las necesidades de su proyecto, en lugar de reinventar la rueda.  Además, GitHub Actions permite a los desarrolladores crear y compartir acciones personalizadas, fomentando un entorno de colaboración en el que la comunidad puede contribuir y beneficiarse de flujos de trabajo reutilizables.

\section{GoRace}
GoRace\cite{gorace} es una suite de aplicaciones web que permite crear automáticamente una solución web a medida para gamificar cualquier dominio. Se integra fácilmente con los sistemas, herramientas o sitios de una empresa, negocio, entorno educativo, etc. y aprovecha las actividades que realizan los participantes para proporcionarles una experiencia única que impulsa el compromiso, la motivación y el éxito de los resultados finales de acuerdo con los objetivos empresariales.

Inspirado en las dinámicas de juego, mecánicas y componentes de conocidos videojuegos, GoRace involucra a los participantes en una experiencia de gamificación basada en la narrativa. En una experiencia GoRace, los participantes se trasladan a un mundo virtual basado en la época de la mitología griega, donde participarán en una legendaria carrera olímpica. El objetivo de la carrera es alcanzar la inmortalidad llegando el primero a la meta. Para ello, los participantes deben cumplir sus actividades en la vida real para obtener distintas recompensas del juego, como puntos de distancia y monedas virtuales, e interactuar con otros elementos del mundo virtual, como una tienda virtual, para comprar y decidir cuándo y cómo utilizar una amplia variedad de objetos de fantasía que pueden tener efectos positivos o negativos en el progreso de un participante en particular o de todos los participantes, y que permiten a los participantes avanzar en la carrera.

\section{GoRace API}
GoRace se refiere a las herramientas externas a GoRace que se emplean en la estrategia de ludificación como recursos del dominio. Estos recursos deben conectarse con GoRace a través de su \acrshort{API}. Los maestros de gamificación deben adaptar sus recursos de dominio para transmitir, mediante la GoRaceAPI, la información relacionada con las actividades de sus participantes conforme a la estrategia de ludificación. Mediante esta configuración, GoRace se integra con los recursos del dominio, permitiendo así que el dominio se conecte con GoRace para el envío y procesamiento automático de los logros de los participantes en sus actividades de la vida real.
