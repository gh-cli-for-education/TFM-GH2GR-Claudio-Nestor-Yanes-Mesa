\section{Conclusions}
This thesis has demonstrated the significant potential of gamification in enhancing IT education by focusing on the development and integration of the GH2GR program. Gamification has been shown to empower learning in the IT field, offering a more engaging and effective educational experience.

The primary focus of this thesis was the development of GH2GR, a software program designed to bridge Github Classroom, a widely-used platform for IT education, with GoRace, an established and proven gamification platform. GH2GR provides an automated and seamless way to incorporate gamification into IT courses, requiring minimal additional effort from educators. This integration allows professors to continue using their familiar tools while enhancing their courses with the benefits of gamification through GoRace.

The development and implementation of GH2GR have been successful, with the program functioning flawlessly and proving to be a reliable tool for educators. This technical success validates the design and execution of GH2GR, confirming its practicality and utility in real-world educational settings.

Initial feedback from users of GH2GR has been generally favourable, highlighting the positive reception and potential impact of the tool. However, the limited amount of feedback collected thus far precludes definitive conclusions regarding its long-term effectiveness and widespread applicability.

In conclusion, this thesis emphasizes the empowering role of gamification in IT education through the development of GH2GR. The program represents a significant advancement in integrating gamification with existing educational tools, promising to enhance student engagement and learning outcomes.

\section{Future lines of work}
The future of this project is highly influenced by the platforms it integrates, so it is difficult to say what the most interesting future lines of work are without knowing how these two platforms will evolve.

As things stand at the moment, the main line of work seems to be to increase the user base and test the application in more complete races in order to see what the real impact of GH2GR is and to check if it really means a positive benefit for the \acrshort{IT} courses that implement it.

Another interesting line of work would be to adapt it to offer it as a \acrshort{SaaS}, avoiding teachers having to manage their own server. Another interesting twist, if the project were to be of interest to the good people at \acrshort{UCA}, would be to further integrate it with GoRace and perhaps offer it as a default integration where the server is already deployed on the GoRace infrastructure.


