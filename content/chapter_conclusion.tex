\section{Conclusiones}
Esta trabajo ha demostrado el importante potencial de la gamificación para mejorar la enseñanza de las TI centrándose en el desarrollo y la integración del programa GH2GR. Se ha demostrado que la gamificación potencia el aprendizaje en el campo de las TI, ofreciendo una experiencia educativa más atractiva y eficaz.

El objetivo principal de este proyecto fue el desarrollo de GH2GR, un programa de software diseñado para unir Github Classroom, una plataforma ampliamente utilizada para la educación en TI, con GoRace, una plataforma de gamificación establecida y probada. GH2GR proporciona una forma automatizada y fluida de incorporar la gamificación a los cursos de TI, requiriendo un mínimo esfuerzo adicional por parte de los educadores. Esta integración permite a los profesores seguir utilizando sus herramientas habituales mientras mejoran sus cursos con los beneficios de la gamificación a través de GoRace.

El desarrollo y la implantación de GH2GR han sido un éxito, ya que el programa ha funcionado a la perfección y ha demostrado ser una herramienta fiable para los educadores. Este éxito técnico valida el diseño y la ejecución de GH2GR, confirmando su practicidad y utilidad en entornos educativos del mundo real.

Los primeros comentarios de los usuarios de GH2GR han sido en general favorables, lo que pone de manifiesto la buena acogida y el impacto potencial de la herramienta. Sin embargo, el escaso número de comentarios recogidos hasta la fecha impide extraer conclusiones definitivas sobre su eficacia a largo plazo y su aplicabilidad generalizada.

En conclusión, esta trabajo pone de relieve el papel potenciador de la gamificación en la enseñanza de las TI a través del desarrollo de GH2GR. El programa representa un avance significativo en la integración de la gamificación con las herramientas educativas existentes, y promete mejorar el compromiso de los estudiantes y los resultados del aprendizaje.

\section{Líneas futuras}
El futuro de este proyecto está altamente influenciado por las plataformas que integra, por lo que es complicado dictaminar cuáles son las líneas de trabajo futuras más interesantes sin saber cómo evolucionarán estas dos plataformas.

Tal y como están las cosas ahora mismo, la línea de trabajo principal parece ser la de aumentar la base de usuarios y poner a prueba la aplicación en carreras más completas para así poder ver cuál es el impacto real de GH2GR y comprobar si de verdad supone un beneficio positivo para los cursos de \acrshort{TI} que lo implementen.

Otra línea de trabajo interesante sería adaptarlo para ofrecerlo como un \acrshort{SaaS}, evitando que el profesorado tenga que gestionar su propio servidor. Otro giro interesante, si el proyecto llegara a interesar a las buenas gentes de la \acrshort{UCA}, sería una mayor integración con GoRace y quizás ofrecerlo como una integración por defecto en la que el servidor ya esté desplegado en la infraestructura de GoRace.
